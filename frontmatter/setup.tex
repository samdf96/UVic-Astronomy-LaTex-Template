%*****************************************************************
% SETUP
%**********************************************************************
\newcommand{\yourname}{A Hopeful Graduate: Samuel Fielder}
\newcommand{\thesistitle}{An Example Thesis Made With \LaTeX \ for Astronomy Students}
% one needs to make adjustments for a Master's Thesis
\newcommand\nameanddegrees{%
\yourname\\
B.Sc., University of WhoKnowsWhere, 2053\\
M.Sc., University of AnotherOne, 2054}
\newcommand\panel{%
\\\panelist{Dr. R.\ Supervisor Main}{Supervisor}{Department of Same As Candidate}
\\\panelist{Dr. M.\ Member One}{Departmental Member}{Department of Same As Candidate}
\\\panelist{Dr. Member Two}{Departmental Member}{Department of Same As Candidate}
\\\panelist{Dr. Outside Member}{Outside Member}{Department of Not Same As Candidate}}
%\HRule\\\panelist{Dr. \Ans{Current Unknown}}{Additional Member}{Department of \Ans{Current Unknown Department}}}

%*************************************************************************************************
% SET INITIAL STYLES
%*************************************************************************************************
\newcommand\tpbreak{\\[\baselineskip]} % titlepage break

\newpage
\thispagestyle{empty} % suppress numbers on the first page

% setting header and footer values
\pagestyle{myheadings}
\pagenumbering{roman}
% We use this in addition to the default \LaTeX page configuration routines
% because we have no way of saying \thispagestyle after the glossary and bibliography starts.
\fancypagestyle{plain}{%
\fancyhf{}
\fancyhead[R]{\thepage}
\renewcommand{\headrulewidth}{0pt}
\renewcommand{\footrulewidth}{0pt}
}

%*************************************************************************************************
% TITLE PAGES
%*************************************************************************************************
\input frontmatter/titlepage % title page
\input frontmatter/committee % supervisory commitee

%*************************************************************************************************
% ABSTACT
%*************************************************************************************************
\input frontmatter/abstract

\renewcommand{\contentsname}{\large\textbf{Table of Contents}}
\TOCadd{Table of Contents}\tableofcontents
\renewcommand{\listtablename}{\large\textbf{List of Tables}}
\TOCadd{List of Tables}\listoftables
\setcounter{lofdepth}{2}
\renewcommand{\listfigurename}{\large\textbf{List of Figures}}
\TOCadd{List of Figures}\listoffigures

%*************************************************************************************************
% GLOSSARY
%*************************************************************************************************
% the following lines need to be uncommented if you are using a glossary
% see https://github.com/PAGSA/UVic-latex-thesis-template for glossary implementation
% \newpage
% \glsaddall     
% \printglossaries 
% \printglossary[title={List of Symbols}] %Use this instead of printglossaries if you want a different title

%*************************************************************************************************
% ACKNOWLEDGEMENTS
%*************************************************************************************************
\input frontmatter/acknowledgements

%*************************************************************************************************
% DEDICATIONS
%*************************************************************************************************
\input frontmatter/dedications

%*************************************************************************************************
% HEADER AND FOOTER SETUP
%*************************************************************************************************
\newpage
\pagestyle{myheadings}
\pagenumbering{arabic}
% We use this in addition to the default LaTeX page configuration routines
% because we have no way of saying \thispagestyle after the bibliography starts.
\fancypagestyle{plain}{%
\fancyhf{}
\fancyhead[R]{\ifnum\thepage=1\relax\else\thepage\fi}
\renewcommand{\headrulewidth}{0pt}
\renewcommand{\footrulewidth}{0pt}
}
