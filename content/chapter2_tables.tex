\startchapter{Tables}
\label{sec: tables}

Similarly with other topics in \LaTeX, there are a multitude of different ways to achieve the same output. As with most basic implementations, Overleaf has fantastic resources. The following link is a great place to start for basic and slightly more advanced table implementations: \url{https://www.overleaf.com/learn/latex/Tables}.

\section{Comprehensive Tables}
\label{sec: tables-comprehensive}

This section will cover what I believe is a fantastic way to implement basic and complex tables. This comes from my experience in working with the \texttt{deluxetable} environment in AASTex, and the related \texttt{aastex} package. Through the process of converting my paper content into my thesis template, I found that the \texttt{deluxetable} environment was not compatible with the University's thesis template, at least not to the extend that I needed it to be (the only change I could not make was the automated setting for the fontsize of the table).

Because of this, I searched for a more basic implementation in some other packages and workflows. While the generic \texttt{tabular} environment is a great place to start, it does not account for the complex table implementations that I needed (tables with merged cells, tables that spilled over pages, landscape tables, etc.).

The following is a list of the packages that allowed me the most flexibility in creating tables that I needed for my thesis. These are in order of the most basic to the most complex:

\subsection{Basis Packages}
\label{sec: tables-basis-packages}

These are what I describe as basis packages, which add in the ability to create tables that are more complex than the basic \texttt{tabular} environment, but are not as complex as the packages that will be discussed later in this section, which essentially have the abilities to \emph{merge} these environments together under one umbrella environment.


\begin{itemize}
    \item \texttt{longtable} - Allows the ability for tables to continue onto the next page. This is great for tables that are too long to fit on a single page.
    \item \texttt{tabularx} - Allows the column designator \texttt{x} to be used, which will automatically adjust the column width in order for the table to fill the declared width of the environment.
    \item \texttt{threeparttable} - Provides a scheme for tables that have a structured note section after the caption.
    \item \texttt{booktabs} - Provides some additional commands to enhance the quality of tables.
    \item \texttt{caption} - Provides the ability to customize the caption of the table. Specifically useful for tables that are too long to fit on a single page (continued captions).
    \item \texttt{multirow} - Provides the ability to merge cells in the row direction.
    \item \texttt{pdflscape} - Provides the ability to create landscape tables.
\end{itemize}

\subsection{Extension Packages}
\label{sec: tables-extension-packages}

As mentioned above, these packages are extensions of the basic packages, and allow for more complex table implementations. These are what made all the different packages work together in a seamless way.

\begin{itemize}
    \item \texttt{threeparttablex} - This package extends the \texttt{threeparttable} package to work with tables created using the \texttt{longtable} package.
    \item \texttt{xltabular} - This package extends the \texttt{longtable} and \texttt{tabularx} packages to work together. Introduces the \texttt{xltabular} environment.
\end{itemize}

\section{Example Tables}
\label{sec: tables-example}

Here is a regular \texttt{tabular} environment, modified by the \texttt{booktabs} package. This is a great way to start creating tables, as it is simple and easy to understand. The \texttt{booktabs} package provides some additional commands to enhance the quality of tables such as the \texttt{toprule}, \texttt{midrule}, and \texttt{bottomrule} commands.

The following is the templated code for the table:

\lstinputlisting[language=TeX]{content/chapter2_tables_simpletable.tex}

The compiled from the above templated code is the following:

\begin{table}[h!]
    \centering
    \begin{tabular}{cccccccc}
        \toprule
        {$m$} & {$M_\odot$} & {$R_\odot$} & {$L_\odot$} & {$A_\nu$} & {$A_m$} & {$\varphi(m)$} & {$\varphi(a)$} \\ \midrule
        1  & 16.128 & +8.872 & 16.128 & 1.402 & 1.373 & -146.6 & -137.6 \\
        2  & 3.442  & -2.509 & 3.442  & 0.299 & 0.343 & 133.2  & 152.4  \\ \midrule
        3  & 1.826  & -0.363 & 1.826  & 0.159 & 0.119 & 168.5  & -161.1 \\
        4  & 0.993  & -0.429 & 0.993  & 0.086 & 0.08  & 25.6   & 90     \\ \bottomrule
    \end{tabular}
\end{table}

Extending this to use a \texttt{threeparttable} environment, including a caption and notes, would be the following:

\lstinputlisting[language=TeX]{content/chapter2_tables_threeparttable.tex}

The compiled from the above templated code is the following:

\begin{table}[h!]
    \centering
    \begin{threeparttable}
        \caption{Example Table with ThreePartTable and Caption
        \label{tab : example-table-threeparttable }}
        \begin{tabular}{cccccccc}
            \toprule
            {$m$} & {$M_\odot$} & {$R_\odot$} & {$L_\odot$} & {$A_\nu$} & {$A_m$} & {$\varphi(m)$} & {$\varphi(a)$} \\ \midrule
            1  & 16.128 & +8.872 & 16.128 & 1.402 & 1.373 & -146.6 & -137.6 \\
            2  & 3.442  & -2.509 & 3.442  & 0.299 & 0.343 & 133.2  & 152.4  \\ \midrule
            3  & 1.826  & -0.363 & 1.826  & 0.159 & 0.119 & 168.5  & -161.1 \\
            4  & 0.993  & -0.429 & 0.993  & 0.086 & 0.08  & 25.6   & 90     \\ \bottomrule
        \end{tabular}
        %
        \begin{tablenotes}[flushleft]
            \footnotesize
            \item \textbf{First Note:} This is a note.
            \item \textit{Second Notes:} This is also a note.
        \end{tablenotes}
    \end{threeparttable}
\end{table}

Expanding now to use the \texttt{tabularx} environment for automatic re-sizing of the columns. For reference, in this example, all columns but the first use the \texttt{X} designation. The following code would be used:

\lstinputlisting[language=TeX]{content/chapter2_tables_tabularx.tex}

The compiled from the above templated code is the following:

\begin{table}[h!]
    \centering
    \begin{threeparttable}
        \caption{Example Table with ThreePartTable and Caption
        \label{tab: example-tabularx}}
        \begin{tabularx}{\textwidth}{cXXXXXXX}
            \toprule
            {$m$} & {$M_\odot$} & {$R_\odot$} & {$L_\odot$} & {$A_\nu$} & {$A_m$} & {$\varphi(m)$} & {$\varphi(a)$} \\ \midrule
            1  & 16.128 & +8.872 & 16.128 & 1.402 & 1.373 & -146.6 & -137.6 \\
            2  & 3.442  & -2.509 & 3.442  & 0.299 & 0.343 & 133.2  & 152.4  \\ \midrule
            3  & 1.826  & -0.363 & 1.826  & 0.159 & 0.119 & 168.5  & -161.1 \\
            4  & 0.993  & -0.429 & 0.993  & 0.086 & 0.08  & 25.6   & 90     \\ \bottomrule
        \end{tabularx}
        %
        \begin{tablenotes}[flushleft]
            \footnotesize
            \item First Note: This is a note.
            \item Second Notes: This is also a note.
        \end{tablenotes}
    \end{threeparttable}
\end{table}

The last example will be a combination of all the above packages, including the \texttt{longtable} package to allow the table to continue onto the next page, the \texttt{multirow} package to merge cells in the row direction, and the \texttt{pdflscape} package to create a landscape table. The following code will be used (data and header information have been selectively trimmed for readability):

\lstinputlisting[language=TeX]{content/chapter2_tables_complex_trimmed.txt}

\begin{landscape}
    \begin{ThreePartTable}
        \begin{TableNotes}[flushleft, para]
            \footnotesize
            $^{\text{a}}$ Properties of the Gaussian fit to the ALMA emission: peak flux, integrated flux, major and minor axes of the FWHM, and position angle of the FWHM. \\
            $^{\text{b}}$ Properties of the deconvolved Gaussian fit: major and minor axes of the FWHM, and position angle of the FWHM. Unresolved sources are indicated by values of -1.
        \end{TableNotes}
        \centering
        \scriptsize
        \begin{xltabular}{\linewidth}{Xccccccccccccccc}
            \caption{Example Table with ThreePartTable, LongTable, Caption, Notes, Landscape\label{tab:observed-properties}}\\
            \toprule
            Scr &  R.A.  &  Decl. & Pk$^{\text{a}}$ & Pk$_{\text{err}}$$^{\text{a}}$ & Tot$^{\text{a}}$ & Tot$_{\text{err}}$$^{\text{a}}$ & FWHM$_{\text{a}}$$^{\text{a}}$ & FWHM$_{\text{b}}$$^{\text{a}}$ &  P.A.$^{\text{a}}$ & \multicolumn{2}{c}{FWHM$_{\text{a,d}}$$^{\text{b}}$ (arcsec)} & \multicolumn{2}{c}{FWHM$_{\text{b,d}}$$^{\text{b}}$ (arcsec)} & \multicolumn{2}{c}{P.A.$_{\text{d}}$$^{\text{b}}$ (deg)}\\
            & (J2000) & (J2000) & \multicolumn{2}{c}{($\text{mJy}~\text{beam}^{-1}$)} & \multicolumn{2}{c}{($\text{mJy}$)} & \multicolumn{2}{c}{(arcsec)} & (deg) & fit & err & fit & err & fit & err \\
            \midrule
            \endfirsthead
            
            \caption[]{(continued from previous page)} \\
            \toprule
            Scr &  R.A.  &  Decl. & Pk$^{\text{a}}$ & Pk$_{\text{err}}$$^{\text{a}}$ & Tot$^{\text{a}}$ & Tot$_{\text{err}}$$^{\text{a}}$ & FWHM$_{\text{a}}$$^{\text{a}}$ & FWHM$_{\text{b}}$$^{\text{a}}$ &  P.A.$^{\text{a}}$ & \multicolumn{2}{c}{FWHM$_{\text{a,d}}$$^{\text{b}}$ (arcsec)} & \multicolumn{2}{c}{FWHM$_{\text{b,d}}$$^{\text{b}}$ (arcsec)} & \multicolumn{2}{c}{P.A.$_{\text{d}}$$^{\text{b}}$ (deg)}\\
            & (J2000) & (J2000) & \multicolumn{2}{c}{($\text{mJy}~\text{beam}^{-1}$)} & \multicolumn{2}{c}{($\text{mJy}$)} & \multicolumn{2}{c}{(arcsec)} & (deg) & fit & err & fit & err & fit & err \\
            \midrule
            \endhead
            
            \bottomrule
            \multicolumn{16}{r}{\footnotesize to be continued on the next page}
            \endfoot
            
            \bottomrule
            \insertTableNotes
            \endlastfoot
            
            1  &  05h46m06.01s &  -00d09m32.70s &  0.36 &    0.08 &  4.60 &     1.03 & 5.471 & 4.622 &  58.8 &  5.26 &    1.22 &  4.43 &    1.07 &   58 &      53 \\
            2  &  05h46m07.26s &  -00d13m30.27s &  9.48 &    0.16 & 12.22 &     0.33 & 1.849 & 1.511 &  71.2 &  0.84 &    0.09 &  0.74 &    0.08 &   65 &      66 \\
            3  &  05h46m07.33s &  -00d13m43.49s & 31.37 &    0.16 & 36.88 &     0.31 & 1.779 & 1.433 &  70.1 &  0.68 &    0.03 &  0.56 &    0.03 &   57 &      12 \\
            4  &  05h46m07.51s &  -00d13m54.79s &  0.45 &    0.16 &  1.36 &     0.63 & 2.978 & 2.177 & 162.1 &  2.67 &    1.19 &  1.42 &    1.04 &  162 &      42 \\
            5  &  05h46m07.53s &  -00d11m49.22s &  0.97 &    0.14 &  5.25 &     0.90 & 4.345 & 2.694 & 152.4 &  4.14 &    0.74 &  2.14 &    0.49 &  153 &      11 \\
            6  &  05h46m07.73s &  -00d12m21.27s & 14.37 &    0.17 & 21.73 &     0.38 & 1.977 & 1.658 &  82.8 &  1.15 &    0.06 &  0.94 &    0.06 &  110 &      12 \\
            7  &  05h46m07.84s &  -00d09m59.61s &  6.45 &    0.11 &  7.59 &     0.21 & 1.726 & 1.361 &  65.2 &  0.81 &    0.07 &  0.36 &    0.10 &   62 &       8 \\
            8  &  05h46m07.86s &  -00d10m01.33s &  2.74 &    0.11 &  3.44 &     0.23 & 1.754 & 1.427 &  63.0 &  0.88 &    0.17 &  0.56 &    0.23 &   57 &      48 \\
            9  &  05h46m08.42s &  -00d10m01.03s &  0.86 &    0.09 &  5.53 &     0.69 & 4.752 & 2.702 &  38.9 &  4.52 &    0.60 &  2.33 &    0.34 &   38 &       8 \\
            10 &  05h46m08.49s &  -00d10m03.10s &  8.13 &    0.10 &  7.55 &     0.17 & 1.444 & 1.284 &  83.2 & -1.00 &   -1.00 & -1.00 &   -1.00 &   -1 &      -1 \\
            11 &  05h46m08.92s &  -00d09m56.11s &  2.07 &    0.11 &  2.28 &     0.20 & 1.576 & 1.392 &  73.8 &  0.51 &    0.31 &  0.36 &    0.21 &  129 &      73 \\
            12 &  05h46m10.04s &  -00d12m16.83s & 39.04 &    0.15 & 40.36 &     0.28 & 1.657 & 1.352 &  72.0 &  0.31 &    0.03 &  0.19 &    0.09 &  165 &      24 \\
            13 &  05h46m13.13s &  -00d06m04.94s &  9.41 &    0.14 & 11.18 &     0.27 & 1.655 & 1.379 &  67.3 &  0.71 &    0.07 &  0.50 &    0.08 &   60 &      19 \\
            14 &  05h46m14.20s &  -00d05m26.71s &  0.51 &    0.13 &  0.54 &     0.24 & 1.704 & 1.177 &  80.5 & -1.00 &   -1.00 & -1.00 &   -1.00 &   -1 &      -1 \\
            15 &  05h46m27.91s &  -00d00m52.11s & 65.62 &    0.14 & 73.72 &     0.27 & 1.662 & 1.409 &  74.0 &  0.52 &    0.02 &  0.49 &    0.02 &   43 &      26 \\
            16 &  05h46m28.34s &  +00d19m49.18s &  1.47 &    0.14 &  1.79 &     0.28 & 1.816 & 1.387 &  78.5 &  0.93 &    0.36 &  0.42 &    0.27 &   77 &      82 \\
            17 &  05h46m28.61s &  +00d20m58.08s &  0.50 &    0.13 &  0.49 &     0.23 & 1.675 & 1.212 & 127.5 &  0.00 &    1.59 &  0.00 &    0.60 &   -1 &      -1 \\
            18 &  05h46m30.91s &  -00d02m35.07s &  7.55 &    0.15 & 17.54 &     0.47 & 2.456 & 1.972 &  63.7 &  1.89 &    0.07 &  1.45 &    0.06 &   58 &       7 \\
            19 &  05h46m31.09s &  -00d02m32.95s & 16.15 &    0.15 & 25.05 &     0.35 & 1.862 & 1.736 & 151.5 &  1.31 &    0.03 &  0.73 &    0.05 &  160 &       5 \\
            20 &  05h46m43.12s &  +00d00m52.47s &  1.70 &    0.13 &  2.40 &     0.28 & 1.759 & 1.619 &   2.1 &  1.15 &    0.33 &  0.56 &    0.42 &  169 &      36 \\
            21 &  05h46m46.52s &  +00d00m16.09s &  1.00 &    0.12 &  1.04 &     0.22 & 1.532 & 1.370 &  63.8 &  0.00 &    1.05 &  0.00 &    0.50 &   -1 &      -1 \\
            22 &  05h46m47.03s &  +00d00m27.20s &  1.96 &    0.13 &  3.55 &     0.34 & 2.212 & 1.652 & 109.8 &  1.69 &    0.25 &  0.83 &    0.30 &  118 &      13 \\
            23 &  05h46m47.43s &  +00d00m23.24s &  1.10 &    0.09 & 12.48 &     1.10 & 5.314 & 4.321 &  37.9 &  5.11 &    0.47 &  4.09 &    0.39 &   36 &      22 \\
            24 &  05h46m47.51s &  +00d00m29.50s &  0.85 &    0.10 &  7.75 &     1.04 & 5.548 & 3.319 &   9.5 &  5.38 &    0.75 &  2.97 &    0.45 &    9 &       9 \\
            25 &  05h46m47.69s &  +00d00m25.02s &  5.38 &    0.13 &  7.22 &     0.27 & 1.810 & 1.496 &  53.6 &  1.01 &    0.11 &  0.64 &    0.13 &   37 &      15 \\
            26 &  05h46m47.97s &  +00d01m41.80s &  1.10 &    0.13 &  2.55 &     0.42 & 2.665 & 1.684 &  38.5 &  2.24 &    0.48 &  0.99 &    0.43 &   34 &      15 \\
            27 &  05h46m57.30s &  +00d23m57.94s &  3.39 &    0.16 &  4.25 &     0.33 & 1.666 & 1.417 &  72.1 &  0.83 &    0.21 &  0.55 &    0.34 &   61 &      72 \\
            28 &  05h47m00.92s &  +00d26m21.98s &  2.65 &    0.16 &  9.34 &     0.71 & 2.782 & 2.363 & 161.4 &  2.46 &    0.21 &  1.87 &    0.19 &  163 &      15 \\
            29 &  05h47m01.31s &  +00d26m23.09s &  4.91 &    0.17 &  6.96 &     0.37 & 1.784 & 1.484 &  92.6 &  1.05 &    0.12 &  0.72 &    0.13 &   99 &      21 \\
            30 &  05h47m10.61s &  +00d21m13.78s & 12.17 &    0.16 & 16.31 &     0.35 & 1.755 & 1.491 &  83.6 &  0.93 &    0.06 &  0.70 &    0.06 &   94 &      12 \\
            31 &  05h47m15.95s &  +00d21m22.89s &  2.02 &    0.13 &  2.70 &     0.28 & 1.805 & 1.553 &  89.0 &  0.92 &    0.29 &  0.75 &    0.51 &  111 &      77 \\
            32 &  05h47m24.84s &  +00d20m58.98s & 16.42 &    0.15 & 51.09 &     0.60 & 2.773 & 2.400 & 133.8 &  2.39 &    0.04 &  1.84 &    0.04 &  144 &       3 \\
            33 &  05h47m32.45s &  +00d20m21.60s &  5.56 &    0.15 &  7.19 &     0.30 & 1.742 & 1.502 &  98.7 &  0.91 &    0.14 &  0.60 &    0.20 &  129 &      21 \\
            34 &  05h47m36.56s &  +00d20m05.89s &  8.36 &    0.15 & 10.28 &     0.29 & 1.651 & 1.508 &  79.1 &  0.75 &    0.07 &  0.58 &    0.12 &  173 &      35 \\
            34 &  05h47m36.56s &  +00d20m05.89s &  8.36 &    0.15 & 10.28 &     0.29 & 1.651 & 1.508 &  79.1 &  0.75 &    0.07 &  0.58 &    0.12 &  173 &      35 \\
            34 &  05h47m36.56s &  +00d20m05.89s &  8.36 &    0.15 & 10.28 &     0.29 & 1.651 & 1.508 &  79.1 &  0.75 &    0.07 &  0.58 &    0.12 &  173 &      35 \\
            34 &  05h47m36.56s &  +00d20m05.89s &  8.36 &    0.15 & 10.28 &     0.29 & 1.651 & 1.508 &  79.1 &  0.75 &    0.07 &  0.58 &    0.12 &  173 &      35 \\
            34 &  05h47m36.56s &  +00d20m05.89s &  8.36 &    0.15 & 10.28 &     0.29 & 1.651 & 1.508 &  79.1 &  0.75 &    0.07 &  0.58 &    0.12 &  173 &      35 \\
            34 &  05h47m36.56s &  +00d20m05.89s &  8.36 &    0.15 & 10.28 &     0.29 & 1.651 & 1.508 &  79.1 &  0.75 &    0.07 &  0.58 &    0.12 &  173 &      35 \\
        \end{xltabular}
    \end{ThreePartTable}
\end{landscape}